% !Mode:: "TeX:UTF-8"
\chapter{分支预测针对解耦前端的设计}

本章首先介绍前端解耦的含义,以及为什么需要对前端解耦,之后详细介绍解耦前端的设计中分支预测相关部分的设计

\section{香山第一版耦合前端简介}

在香山第一版的设计中,而我们把分支预测和取指统称为流水线的前端。相对的以译码为边界界,译码和译码之后的流水级我们统称为后端。而在第一版的前端设计里,取指单元和分支预测是耦合在一起的,只有当分支预测和指令缓存的访问都完成之后,当前流水级才能够流向下一流水级,这会导致分支预测和指令缓存的访问互相阻塞。当访问指令缓存发现miss时,即使分支预测已经完成,也仍然需要停下来等待指令缓存从下级缓存中得到回填的数据;同样的当分支预测需要恢复,冲刷流水线时,即使指令缓存已经准备好被访问了,由于流水线被冲刷,暂时没有新的访问请求,也只能够等待分支预测执行。这种相互掣肘的耦合关系通过解耦,可以将大量的气泡消除,即将分支预测流水线和取指流水线分离,中间由一个队列连接,这个队列就是FTQ (Fetch Target Queue)。

\section{通过FTQ实现前端解耦}

通过将分支预测流水线和取指流水线解耦开,通过FTQ连接,如图2.1所示,就能够将分支预测和取指单元变成类似于生产者和消费者的关系:分支预测作为生产者,负责不断地预测当前pc下一拍的取指pc,而不用管指令缓存是否miss,只需要将相关的取指请求放入FTQ之中。而取指单元作为消费者,负责不断地从FTQ中顺序取出取指的请求,访问指令缓存得到指令码,将它传给后端。由于在指令缓存miss时,分支预测仍旧会不断地往FTQ中送入取指请求,因此当分支预测冲刷流水线时,取指单元也不用等待,可以继续取出FTQ中之前存储的取指请求,继续访问指令缓存取指。这样一来就能够减少前端很多不必要的气泡,前端的供指效率能够得到提高,这也会对整体架构的性能有一定的正面影响

介绍FTQ的具体设计实现

\section{本章小结}

