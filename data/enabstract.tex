% !Mode:: "TeX:UTF-8"

%%% 英文摘要填写部分 %%%

\enabstract{

    In the field of modern high-performance processor design, the Branch Prediction Unit (BPU) is an essential part of a processor architecture, in order to pursue higher performance, processors are often designed in the direction of high frequency and deep pipeline, High frequency limits the complexity of logic per cycle, and deep pipelines incur higher penalties for branch misprediction. There is a lot of research on branch prediction algorithms. For example, the TAGE predictor is one of the state-of-the-art branch predictors currently publicly designed. However, most of the papers on branch prediction research are based on simulator implementation, and there are more constraints and physical implementation details to be considered in the real hardware design process.

    Based on the first version of the Yanqihu architecture of Xiangshan RISC-V open source high-performance processor, this thesis designs the Branch Prediction Unit in the second version of the Nanhu architecture. The Nanhu architecture is a new architecture that is reconstructed and upgraded on the Yanqihu architecture.The new Branch Prediction Unit is designed to decouple the Branch Prediction Unit and the Instruction Fetch Units for higher performance, while reducing critical path latency by reducing the branch prediction width for higher frequencies. The following is the main research content of this thesis:

    (1) The Branch Prediction Unit is decoupled from the Instruction Fetch Unit by adding an FTQ (Fetch Target Queue), which is used to save instruction fetch requests and branch prediction results, and control branch prediction recovery and update.

    (2) By improving BTB (Branch Target Buffer), the basic unit of instruction fetch is changed from the 32 Bytes-aligned fixed-length instruction block of Yanqihu architecture to the variable-length instruction block with Fetch Block as the unit, this limits the maximum number of branch instructions per fetch, and also modified all predictors to use Fetch Block as base prediction unit.

    The main innovation of this thesis is to propose and implement a branch prediction architecture suitable for decoupling the frontend, and a branch prediction architecture that limits the width of branch prediction, which can bring both performance and frequency improvements to the processor. It is the first implementation in the field of RISC-V open source processor research. 

    Evaluate overall architecture performance by performing behavioral-level simulations with Verilator, and using checkpointing techniques to run snippets of SPEC 2006, with performance counters for statistics and analysis. Compared with the first version. 

    In summary, this thesis studies the branch prediction architecture of RISC-V high-performance processors, as well as the specific problems in the physical implementation process, and proposes and completes a specific implementation of branch prediction. Finally, it is evaluated by behavioral simulation. The experimental results show that compared with the Yanqihu architecture, the frequency of Nanhu can be increased 11\% under the 14nm process, and the IPC of some SPEC 2006 test programs is increased by an average of 31\%. 

}

\enkeywords{RISC-V;~~high performance processor;~~CPU;~~decoupled frontend;~~branch prediction}

\enpageheader{Research on Branch Prediction Component Based on RISC-V High Performance Processor}

% 使用makeenabstract生成英文摘要
\makeenabstract