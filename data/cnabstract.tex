% !Mode:: "TeX:UTF-8"

%%% 中文摘要填写部分 %%%

\cabstract{

    在现代的高性能处理器设计领域中,分支预测部件 (Branch Prediction Unit, BPU)是一个处理器架构中必不可少的一部分,处理器为了追求更高的性能往往会往高频率,深流水线的方向设计,高频率就限制了每周期内逻辑的复杂度,而深流水线则会带来更高的分支预测错误惩罚。关于分支预测算法的研究有很多,还有 CBP (Championship Branch Prediction) 这样的分支预测赛事促进分支预测发展。TAGE 预测器是目前公开设计的先进分支预测器之一,曾多次获得 CBP 比赛的冠军。但是分支预测研究的论文大多基于模拟器实现,而真实的硬件设计过程中会有更多的约束和细节需要考虑。

    本文在香山 RISC-V 开源高性能处理器第一版雁栖湖架构的基础上,设计了第二版南湖架构中的分支预测组件,南湖架构是在雁栖湖架构上整体进行重构升级的新架构。新的分支预测旨在实现分支预测和取指单元的解耦以达到更高的性能,同时通过降低分支预测宽度来减少关键路径的延迟,以达到更高的频率。通过添加一个 FTQ (Fetch Target Queue) 将分支预测与取指单元解耦,FTQ 用于保存分支预测结果,控制分支预测恢复和更新。此外,通过改进 BTB (Branch Target Buffer),将取指的基本单位由雁栖湖架构的 32 Byte 对齐的定长指令块改为了以 Fetch Block 为单位的不定长指令块,以此来限制每次取指的分支指令数量,并将所有预测器的基本单位也都修改为使用 Fetch Block 预测。

    在验证方面,使用 Verilator 进行行为级仿真,使用 checkpoint 技术运行 SPEC 2006 的片段以评估整体架构性能,通过性能计数器来统计相关数据并进行分析。

    综上所述,本文研究了 RISC-V 高性能处理器的分支预测架构,以及物理实现过程中的具体问题,提出并完成了一种分支预测的具体实现。最终通过行为级仿真对其进行了评估,实验结果表明相对于雁栖湖架构,在14nm的工艺下,南湖架构的频率能够从1.8Ghz提升到2GHz,运行部分SPEC 2006测试程序IPC平均提升31\%。

}

\ckeywords{RISC-V;~~高性能处理器;~~CPU设计;~~解耦前端;~~分支预测}

\cnpageheader{基于RISC-V高性能处理器的分支预测部件研究}  %页眉标题无需断行

% 使用makecnabstract生成中文摘要
\makecnabstract