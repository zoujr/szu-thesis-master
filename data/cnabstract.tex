% !Mode:: "TeX:UTF-8"

%%% 中文摘要填写部分 %%%

\cabstract{

    在现代的高性能处理器设计领域中,分支预测是一个处理器架构中必不可少的一部分,处理器为了追求更高的性能往往会往高频率,深流水线的方向设计,高频率就限制了每周期内逻辑的复杂度,而深流水线则会使分支预测错误带来的惩罚更高。关于分支预测算法的研究有很多,还有CBP (Championship Branch Prediction) 这样的分支预测赛事促进分支预测发展,TAGE预测器是目前公开设计的先进分支预测器之一,曾多次获得CBP比赛的冠军。但是分支预测研究的论文大多基于模拟器实现,而真实的硬件设计过程中会有更多的约束和细节需要考虑。
    本文基于香山RISC-V开源高性能处理器第一版架构的基础上,设计了第二版分支预测架构,旨在实现解耦前端的设计以达到更高的性能,同时通过减少分支预测宽度来减关键路径的延迟,达到更高的频率。另外还尝试了一些分支预测相关的优化和探索。首先我们通过添加一个FTQ (Fetch Target Queue) 来讲分支预测与取指逻辑解耦,通过FTQ来保存分支预测结果,控制分支预测恢复和更新。此外,通过改进BTB (Branch Target Buffer),我们将取指的基本单位由第一版的32Bytes对齐的定长指令块改为了以fetch block为单位的不定长指令块,以此来限制每次取指的分支指令数量,并将所有预测器的基本单位也都修改为使用fetch block预测。
    通过使用Verilator进行行为级仿真,并使用checkpoint技术来运行SPEC 2006的片段来评估整体架构性能,通过性能计数器来统计(具体的性能提升数据)
    综上所述,本文研究了RISC-V高性能处理器的分支预测架构,以及物理实现过程中的具体问题,提出并完成了一种分支预测的具体实现。最终通过行为级仿真对齐进行了评估,实验结果表明相对于第一版架构,新的设计有xxx的提升。

}

\ckeywords{RISC-V;~~高性能处理器;CPU设计;~~解耦前端;~~分支预测}

\cnpageheader{摘要}  %页眉标题无需断行

% 使用makecnabstract生成中文摘要
\makecnabstract