% !Mode:: "TeX:UTF-8"

\chapter{总结与展望}

\section{总结}

在高性能处理器的分支预测研究领域中,为了提高分支预测准确率,大量的学者提出了各种各样的算法和改进措施,也取得了很好的效果。但是大部分论文中的算法设计都是基于模拟器进行性能分析,并没有考虑到现实物理设计时的限制,而商业处理器的分支预测具体设计大部分细节都不会公开。

而本文选用香山这一RISC-V开源高性能处理器作为平台,在其第一版架构的基础上进行了重构和改进,提出了一种新的分支预测架构,来提高整体频率的同时也能够提升性能,最终作为香山处理器第二版的分支预测组件设计,即将送往流片。相对于第一版来说,在相同的14nm工艺下,使用Design Compiler进行评估,整体的频率能够从第一版的1.8GHz提升到2.0GHz,且在SPEC 2006下整体性能能够提升百分之xx。

\section{展望}

本文提出的分支预测架构在第一版的基础上做出了改进,取得了良好的效果,但是其实现在的设计还有很多不足和待改进的地方。总结以及改进思路如下:

