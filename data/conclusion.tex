% !Mode:: "TeX:UTF-8"

\chapter{总结与展望}

\section{总结}

在高性能处理器的分支预测研究领域中,为了提高分支预测准确率,大量的学者提出了各种各样的算法和改进措施,也取得了很好的效果。但是大部分论文中的算法设计都是基于模拟器进行性能分析,并没有考虑到现实物理设计时的限制,而商业处理器的分支预测具体设计大部分细节都不会公开。

而本文选用香山这一RISC-V开源高性能处理器作为平台,在其第一版架构的分支预测部件基础上进行了重构和改进,提出了一种新分支预测架构的设计实现,其中主要的两个改进就是实现FTB来限制分支预测宽度,已经将分支预测和取指单元的流水线解耦来减少前端流水线的气泡。此外一些分支预测器中也有许多优化逻辑电路延迟的改动。最终作为香山处理器第二版的分支预测组件设计,即将送往流片。相对于第一版来说,总体性能和频率都有一定的提升。在相同的14nm工艺下,使用Design Compiler进行评估,整体的频率能够从第一版的1.8GHz提升到2.0GHz,提升了11\%,且从SPEC 2006下的部分测试程序结果来看,整体IPC性能能够提升17.84\%。

\section{展望}

本文提出的分支预测架构在第一版的基础上做出了改进,取得了良好的效果,但是其实现在的设计还有很多不足和待改进的地方。这里提出一部分的改进思路和未来的研究方向:

\begin{enumerate}
    \item 在开发过程中,如何快速准确的对分支预测部件进行功能评估,一直是一个没有解决的问题,由于分支预测本身就是一个预测性质的功能,即使预测错误也只是会造成性能损失,不会出现功能上的错误,因此香山的差分测试框架难以找到分支预测实现上的bug。如何测试分支预测运行时各种功能是否正常,是一个非常值得深入研究的课题。
    \item 在香山下一版的设计,以及未来的每次迭代中,对频率和性能的要求肯定是越来越高的,因此在第二版的基础上,需要考虑如何能够做到更高的频率,可能分支预测的整体框架还要大幅的改动。
    \item 在第一版的设计中,分支预测里实现了循环预测器,但是在第二版中移除了它,主要是因为第二版使用了非对齐的取指策略,同一条分支可能存在于不同的Fetch Block中,这样对于循环预测器的实现带来了一些问题,循环预测器无法准确的记录某条循环跳转指令的迭代次数。因此之后如何设计实现循环预测器,将其加入到分支预测中,也是一个未来的改进方向。
\end{enumerate}

