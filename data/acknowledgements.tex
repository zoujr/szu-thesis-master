% !Mode:: "TeX:UTF-8"

\newenvironment{theacknowledgements}{\wuhao\song}

\addcontentsline{toc}{chapter}{致谢}%添加到目录中
\chapter*{\centering\xiaosan\hei\bfseries 致\quad 谢}

\begin{theacknowledgements}
	
随着三年硕士研究生生活的落幕,也代表着我多年的求学生活已经到达了尾声,我很庆幸在学生生涯的最后一段时间能够遇到我的导师和同学们。在刚得知自己被录取的那天,去导师实验室和他面谈的情景仍然历历在目;研一和同学们一起上课,为课程作业而发愁的时光也仿佛就在昨天,而转眼三年就已经过去。导师和同学们在学习和生活上给予了我许多的帮助,让我很快适应了研究生忙碌且充实的生活。也很感谢深圳大学,虽然因为疫情和实习原因,我真正待在校园里的时间不长,但我仍然对校园美丽的景色印象深刻,无论早晨傍晚深夜,还是刮风下雨天晴,走在校园中,处处都是不同的风景。这一切都使得这三年将成为我今后不断怀念的一段时光。

首先我最感谢的就是我的导师,蔡晔老师,是他带领我踏入了计算机体系结构的领域,我在本科时学习的是软件专业,后来由于对硬件设计感兴趣而找到了蔡老师,蔡老师三年间在我的学习研究,生活和为人处事等各个方面都给予了我许多的关心和指导,关心我的学习生活情况,为我答疑解惑,指导纠正我研究中的错误与不足,让我从未曾度过一篇论文,到自己参与实习,并成功发表了属于自己的会议论文;从之前对计算机体系结构和硬件设计几乎零基础,到今天有参加大型的工程项目的宝贵的经验,这一切都离不开蔡老师的指导和帮助。老师开阔的学术思维,严谨的学术态度,和对我们严格的要求,也将成为我今后不断勉励自己进步的标准。

此外还要感谢在实习期间,同样给予我许多指导的包云岗老师,唐丹老师,解壁伟老师,郇丹丹老师,赵继业老师,李祖松老师等,他(她)们让我了解了业内顶尖的学者和工程师的风貌,也传授给我许多宝贵的经验。在北京实习期间,在生活和工作上都给予了我极大的关心和照顾,多次跟我探讨我的毕业课题研究方向,引导我一步步地完成自己的工作。同样感谢的还有在实习中所有的同事们,在我入门时耐心的教我如何搭建环境,给我解答问题,在我完成毕业课题和论文的期间也给予了我大量的建议和帮助。

还要感谢同门的毕壹双、甘海洋、罗浩鑫师兄和陈子韵师姐,张发旺,陈仕健和曾智圣同学,还有我的师弟们以及我的室友们,在平时工作中大家是可靠的伙伴,能够一起交流探讨许多有趣的问题和技术,生活中也是亲密的朋友,大家一起学习一起生活,使科研生活增添了许多色彩,也留下了许多令人难忘的回忆。

最后还要感谢我的父母和我的亲人们,一直以来都在默默地支持着我,支持我自己做出的所有决定和选择。即使我远在异国他乡,父母也经常电话中询问我的学习状况,生活费是否够用,告诉我家中一切安好不必担心。无论在外有多辛苦,只要我回到家中,就能够由衷地感到一种宁静与放松。他们对我的支持也能够让我能够全身心投入到学习和工作中。最重要的是感谢他们对我二十多年的苦心栽培,在我身上付出的无数心血,才成就了今天的我。

一个篇章的落幕是另一段新篇章的开始。我即将正式踏入社会,未来充满了未知和挑战,也有无数新的风景与人等着我去发现去结识。此时此刻,心中既有对未来的迷茫,亦含对新生活的向往与好奇。今后的路还很长,我已经准备好迎接新的挑战,无论结果如何,我都会不断努力,不辜负所有曾在我人生道路中帮助过我的人对我的期望。

\end{theacknowledgements}





