% !Mode:: "TeX:UTF-8"

\newenvironment{theacknowledgements}{\wuhao\song}

\addcontentsline{toc}{chapter}{致谢}%添加到目录中
\chapter*{\centering\xiaosan\hei\bfseries 致\quad 谢}

\begin{theacknowledgements}
	
随着三年硕士研究生生活的落幕,也代表着我多年的求学生活已经到达了尾声,我很庆幸在学生生涯的最后一段时间能够遇到我的导师和同学们,他们在学习和生活上给予了我许多的帮助,也很感谢深圳大学,虽然因为疫情和实习原因,我真正待在校园里的时间不长,但美丽的景色,舒适的宿舍,还有导师和同学们,都使得这三年将会成为我今后不断怀念的一段时光。

首先我要感谢我的导师,***老师,是他带领我踏入了计算机体系结构的领域,我在本科时学习的是软件专业,后来由于对硬件设计感兴趣而找到了*老师,*老师在三年间给了我许许多多的指导,让我从之前对计算机体系结构和硬件设计几乎零基础,到今天能够参加大型的工程项目,并积累了许多宝贵的经验。这一切都离不开*老师的指导和帮助。

此外还要感谢在实习期间,同样给予我许多指导的包云岗老师,唐丹老师,解壁伟老师,郇丹丹老师,赵继业老师,李祖松老师,他们让我了解了业内顶尖的学者和工程师的风貌,也传授给我许多宝贵的经验。同样感谢的还有在实习中所有的同学们,在我入门时耐心的给我解答问题,在毕业时也给了我大量的帮助。

还要感谢同门的毕壹双、甘海洋、罗浩鑫师兄和陈子韵师姐,张发旺,陈仕健和曾智圣同学,以及我的室友们,在平时工作中大家是可靠的伙伴,能够一起交流探讨许多有趣的问题和技术,生活中也是亲密的朋友,大家一起聊天聚餐,使科研生活增添了许多色彩。

最后感谢我的父母和我的亲人们,一直以来都在默默地支持着我,支持我所有的选择,无论在外有多辛苦,只要我回到家中,就能够由衷地感到一种宁静与放松。他们对我的支持也能够让我能够全身心投入到学习和工作中。

一个篇章的落幕是另一段新篇章的开始。我即将正式踏入社会,未来充满了未知和挑战,也有无数新的风景与人等着我去发现。抛去对未来的迷茫,我更多的是对新生活的向往与好奇。今后的路还很长,我已经准备好迎接新的挑战,无论结果如何,我都会不断努力,不辜负所有曾在我人生道路中帮助过我的人对我的期望。

\end{theacknowledgements}





